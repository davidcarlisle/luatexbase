% \iffalse meta-comment
%
% Written in 2009, 2010 by Manuel Pégourié-Gonnard and Élie Roux.
%     <mpg@elzevir.fr>
%     <elie.roux@telecom-bretagne.eu>
%
% This work is under the CC0 license.
%
% This work consists of the main source file luatexbase-mcb.dtx
% and the derived files
%    luatexbase-mcb.sty, mcb.lua, luatexbase-mcb.pdf,
%    test-mcb-plain.tex test-mcb-latex.tex
%
% Unpacking:
%    tex luatexbase-mcb.dtx
% Documentation:
%    pdflatex luatexbase-mcb.dtx
%
%<*ignore>
\begingroup
  \def\x{LaTeX2e}%
\expandafter\endgroup
\ifcase 0\ifx\install y1\fi\expandafter
         \ifx\csname processbatchFile\endcsname\relax\else1\fi
         \ifx\fmtname\x\else 1\fi\relax
\else\csname fi\endcsname
%</ignore>
%<*install>
\input docstrip.tex

\keepsilent
\askforoverwritefalse

\let\MetaPrefix\relax

\preamble

Copyright (C) 2009 by Elie Roux <elie.roux@telecom-bretagne.eu>

This work is under the CC0 license.
See source file '\inFileName' for details.

\endpreamble

\let\MetaPrefix\DoubleperCent

\generate{%
  \usedir{tex/luatex/luatexbase}%
  \file{luatexbase-mcb.sty}{\from{luatexbase-mcb.dtx}{texpackage}}%
}

\generate{%
  \usedir{doc/luatex/luatexbase}%
  \file{test-mcb-plain.tex}{\from{luatexbase-mcb.dtx}{testplain}}%
  \file{test-mcb-latex.tex}{\from{luatexbase-mcb.dtx}{testlatex}}%
}

\def\MetaPrefix{-- }

\def\luapostamble{%
  \MetaPrefix^^J%
  \MetaPrefix\space End of File `\outFileName'.%
}

\def\currentpostamble{\luapostamble}%

\generate{%
  \usedir{tex/luatex/luatexbase}%
  \file{mcb.lua}{\from{luatexbase-mcb.dtx}{lua}}%
}

\generate{%
  \usedir{doc/luatex/luatexbase}%
  \file{test-mcb.lua}{\from{luatexbase-mcb.dtx}{testlua}}%
}

\obeyspaces
\Msg{************************************************************************}
\Msg{*}
\Msg{* To finish the installation you have to move the following}
\Msg{* files into a directory searched by TeX:}
\Msg{*}
\Msg{*     luatexbase-mcb.sty mcb.lua}
\Msg{*}
\Msg{* Happy TeXing!}
\Msg{*}
\Msg{************************************************************************}

\endbatchfile
%</install>
%<*ignore>
\fi
%</ignore>
%<*driver>
\documentclass{ltxdoc}
% preamble for dtx documentations of the luatexbase package/bundle.

\usepackage[T1]{fontenc}
\usepackage{lmodern}
\usepackage{geometry}
\usepackage{xspace}
\usepackage[english]{babel}
\usepackage[colorlinks]{hyperref}
\usepackage{bookmark}

% logos
\makeatletter
\newcommand\eTeX{$\m@th\varepsilon$-\TeX}
\newcommand\LuaTeX{Lua\TeX}
\renewcommand\PlainTeX{Plain\thinspace\TeX}
\newcommand\TeXe{\TeX\thinspace82}
\newcommand\TeXLive{\TeX\thinspace Live}
\makeatother

% logos for the lazy me (mpg)
\newcommand\tex{\TeX\xspace}
\newcommand\latex{\LaTeX\xspace}
\newcommand\etex{\eTeX\xspace}
\newcommand\luatex{\LuaTeX\xspace}
\newcommand\texe{\TeXe\xspace}
\newcommand\texlive{\TeXLive\xspace}
\newcommand\plaintex{\PlainTeX\xspace}

% special elements of text
\newcommand\file{\nolinkurl}
\newcommand\email[1]{\href{mailto:#1}{#1}}
\newcommand\pk{\textsf}
\newcommand\cmdname{\texttt}

% for hyperref
\pdfstringdefDisableCommands{%
  \def\cs#1{\@backslashchar #1}%
  }

% easy verbatim
\MakeShortVerb\|

\begin{document}
  \DocInput{luatexbase-mcb.dtx}%
\end{document}
%</driver>
% \fi
%
% \CheckSum{0}
%
% \CharacterTable
%  {Upper-case    \A\B\C\D\E\F\G\H\I\J\K\L\M\N\O\P\Q\R\S\T\U\V\W\X\Y\Z
%   Lower-case    \a\b\c\d\e\f\g\h\i\j\k\l\m\n\o\p\q\r\s\t\u\v\w\x\y\z
%   Digits        \0\1\2\3\4\5\6\7\8\9
%   Exclamation   \!     Double quote  \"     Hash (number) \#
%   Dollar        \$     Percent       \%     Ampersand     \&
%   Acute accent  \'     Left paren    \(     Right paren   \)
%   Asterisk      \*     Plus          \+     Comma         \,
%   Minus         \-     Point         \.     Solidus       \/
%   Colon         \:     Semicolon     \;     Less than     \<
%   Equals        \=     Greater than  \>     Question mark \?
%   Commercial at \@     Left bracket  \[     Backslash     \\
%   Right bracket \]     Circumflex    \^     Underscore    \_
%   Grave accent  \`     Left brace    \{     Vertical bar  \|
%   Right brace   \}     Tilde         \~}
%
% \title{The \textsf{luatexbase-mcb} package}
% \date{2010/05/27 v0.2a}
% \author{%
%   Manuel P\'egouri\'e-Gonnard \\ \email{mpg@elzevir.fr} \and
%   \'Elie Roux \\ \email{elie.roux@telecom-bretagne.eu}}
%
% \maketitle
%
% \begin{abstract}
% The primary feature of this package is to allow many functions to be
% registered in the same callback. Depending of the type of the callback, the
% functions will be combined in some way when the callback is called. Functions
% are provided for addition and removal of individual functions from a
% callback's list, with a priority system.\par
% Additionally, you can create new callbacks that will be handled the same way
% as predefined callbacks, except that they must be called explicitly.
% \end{abstract}
%
% \tableofcontents
%
% \section{Documentation}
%
% Before we start, let me mention that test files are provided (they should be
% in the same directory as this PDF file). You can have a look at them,
% compile them and have a look at the log, if you want examples of how this
% module works.
%
% \subsection{Managing functions in callbacks}
%
% \luatex provides an extremely interesting feature, named callbacks. It
% allows to call some Lua functions at some points of the \TeX\ algorithm (a
% \emph{callback}), like when \TeX\ breaks likes, puts vertical spaces, etc.
% The \luatex core offers a function called \texttt{callback.register} that
% enables to register a function in a callback.
%
% The problem with |callback.register| is that is registers only one function
% in a callback. This package solves the problem by disabling
% |callback.register| and providing a new interface allowing many functions to
% be registered in a single callback.
%
% The way the functions are combined together depends on
% the type of the callback. There are currently 4 types of callback, depending
% on the calling convention of the functions the callback can hold:
% \begin{description}
%   \item[simple] is for functions that don't return anything: they are called
%     in order, all with the same argument;
%   \item[data] is for functions receiving a piece of data of nay type
%     except node list head (and possibly other arguments) and returning it
%     (possibly modified): the functions are called in order, and each is
%     passed the return value of the previous (and the other arguments
%     untouched, if any). The return value is that of the last function;
%   \item[list] is a specialized variant of \emph{data} for functions
%     filtering node lists. Such functions may return either the head of a
%     modified node list, or the boolean values |true| or |false|. The
%     functions are chained the same way as for \emph{data} except that for
%     the following. If
%     one function returns |false|, then |false| is immediately return and the
%     following functions are \emph{not} called. If one function returns
%     |true|, then the same head is passed to the next function. If all
%     functions return |true|, then |true| is returned, otherwise the return
%     value of the last function not returning |true| is used.
%   \item[first] is for functions with more complex signatures; functions in
%     this type of callback are \emph{not} combined: only the first one
%     (according to priorities) is executed.
% \end{description}
%
% To add a function to a callback, use:
% \begin{verbatim}
% luatexbase.add_to_callback(name, func, description, priority)
% \end{verbatim}
% The first argument is the name of the callback, the second is a function,
% the third one is a string used to identify the function later, and the
% optional priority is a positive integer, representing the rank of the
% function in the list of functions to be executing for this callback. So,
% |1| is the highest priority. If no priority is specified, the function is
% appended to the list, that is, its priority is the one of the last function
% plus one.
%
% The priority system is intended to help resolving conflicts between packages
% competing on the same callback, but it cannot solve every possible issue. If
% two packages request priority |1| on the same callback, then the last one
% loaded will win.
%
% To remove a function from a callback, use:
% \begin{verbatim}
% luatexbase.remove_from_callback(name, description)
% \end{verbatim}
% The first argument must be the name of the callback, and the second one the
% description used when adding the function to this callback. You can also
% remove all functions from a callback at once using
% \begin{verbatim}
% luatexbase.reset_callback(name, make_false)
% \end{verbatim}
% The |make_false| argument is optional. If it is |true| (repeat: |true|, not
% |false|) then the value |false| is registered in the callback, which has a
% special meaning for some callback.
%
% Note that |reset_callback| is very invasive since it removes all functions
% possibly installed by other packages in this callback. So, use it with care
% if there is any chance that another package wants to share this callback
% with you.
%
% When new functions are added at the beginning of the list, other functions
% are shifted down the list. To get the current rank of a function in a
% callback's list, use:
% \begin{verbatim}
% priority = luatexbase.priority_in_callback(name, description)
% \end{verbatim}
% Again, the description is the string used when adding the function. If the
% function identified by this string is not in this callback's list, the
% priority returned is the boolean value |false|.
%
% \subsection{Creating new callbacks}
%
% This package also provides a way to create and call new callbacks, in
% addition to the default \luatex callbacks.
% \begin{verbatim}
% luatexbase.create_callback(name, type, default)
% \end{verbatim}
% The first argument is the callback's name, it must be unique. Then, the type
% goes as explained above, it is given as a string. Finally all user-defined
% callbacks have a default function which must\footnote{You can obviously
% provide a dummy function. If you're doing so often, please tell me, I may
% want to make this argument optional.} be provided as the third
% argument. It will be used when no other function is registered for this
% callback.
%
% Functions are added to and removed from user-defined callbacks just the same
% way as predefined callback, so the previous section still applies. There is
% one difference, however: user-defined callbacks must be called explicitly at
% some point in your code, while predefined callbacks are called automatically
% by \luatex. To do so, use:
% \begin{verbatim}
% luatexbase.call_callback(name, arguments...)
% \end{verbatim}
% The functions registered for this callback (or the default function) will be
% called with |arguments...| as arguments.
%
% \subsubsection{Limitations}
%
% For callbacks of type |first|, our new management system isn't actually
% better than good old |callback.register|. For some of them, is may be
% possible to split them into many callbacks, so that these callbacks can
% accept multiple functions. However, its seems risky and limited in use and
% is therefore nor implemented.
%
% At some point, \pk{luatextra} used to split |open_read_file| that way, but
% support for this was removed. It may be added back (as well as support for
% other split callbacks) if it appears there is an actual need for it.
%
% \subsection{Compatibility}
%
% Some callbacks have a calling convention that varies depending on the
% version of \luatex used. This package \emph{does not} try to track the type
% of the callbacks in every possible version of \luatex. The types are based
% on the last stable beta version (0.60.2 at the time this doc is written).
%
% However, for callbacks that have the same calling convention for every
% version of \luatex, this package should work with the same range of \luatex
% version as other packages in the \pk{luatexbase} bundle (currently, 0.25.4
% to 0.60.2).
%
%    \section{Implementation}
%
%    \subsection{\tex package}
%
%    \begin{macrocode}
%<*texpackage>
%    \end{macrocode}
%
%    \subsubsection{Preliminaries}
%
%    Reload protection, especially for \plaintex.
%
%    \begin{macrocode}
                \csname lltxb@mcb@loaded\endcsname
\expandafter\let\csname lltxb@mcb@loaded\endcsname\endinput
%    \end{macrocode}
%
%    Catcode defenses.
%
%    \begin{macrocode}
\begingroup
  \catcode123 1 % {
  \catcode125 2 % }
  \catcode 35 6 % #
  \toks0{}%
  \def\x{}%
  \def\y#1 #2 {%
    \toks0\expandafter{\the\toks0 \catcode#1 \the\catcode#1}%
    \edef\x{\x \catcode#1 #2}}%
  \y 123 1  % {
  \y 125 2  % }
  \y  35 6  % #
  \y  10 12 % ^^J
  \y  34 12 % "
  \y  36 3  % $ $
  \y  39 12 % '
  \y  40 12 % (
  \y  41 12 % )
  \y  42 12 % *
  \y  43 12 % +
  \y  44 12 % ,
  \y  45 12 % -
  \y  46 12 % .
  \y  47 12 % /
  \y  60 12 % <
  \y  61 12 % =
  \y  64 11 % @ (letter)
  \y  62 12 % >
  \y  95 12 % _ (other)
  \y  96 12 % `
  \edef\y#1{\endgroup\edef#1{\the\toks0\relax}\x}%
\expandafter\y\csname lltxb@mcb@AtEnd\endcsname
%    \end{macrocode}
%
%    Package declaration.
%
%    \begin{macrocode}
\begingroup
  \expandafter\ifx\csname ProvidesPackage\endcsname\relax
    \def\x#1[#2]{\immediate\write16{Package: #1 #2}}
  \else
    \let\x\ProvidesPackage
  \fi
\expandafter\endgroup
\x{luatexbase-mcb}[2010/05/27 v0.2a Callback management for LuaTeX]
%    \end{macrocode}
%
%    Make sure \luatex is used.
%
%    \begin{macrocode}
\begingroup\expandafter\expandafter\expandafter\endgroup
\expandafter\ifx\csname RequirePackage\endcsname\relax
  \input ifluatex.sty
\else
  \RequirePackage{ifluatex}
\fi
\ifluatex\else
  \begingroup
    \expandafter\ifx\csname PackageWarningNoLine\endcsname\relax
      \def\x#1#2{\begingroup\newlinechar10
        \immediate\write16{Package #1 warning: #2}\endgroup}
    \else
      \let\x\PackageWarningNoLine
    \fi
  \expandafter\endgroup
  \x{luatexbase-mcb}{LuaTeX is required for this package. Aborting.}
  \lltxb@mcb@AtEnd
  \expandafter\endinput
\fi
%    \end{macrocode}
%
%    \subsubsection{Load supporting Lua module}
%
%    First load \pk{luatexbase-loader} (hence \pk{luatexbase-compat}), then
%    the supporting Lua module.
%
%    \begin{macrocode}
\begingroup\expandafter\expandafter\expandafter\endgroup
\expandafter\ifx\csname RequirePackage\endcsname\relax
  \input luatexbase-modutils.sty
\else
  \RequirePackage{luatexbase-modutils}
\fi
\luatexbase@directlua{require('luatexbase.mcb')}
%    \end{macrocode}
%
%    That's all folks!
%
%    \begin{macrocode}
\lltxb@mcb@AtEnd
%</texpackage>
%    \end{macrocode}
%
%    \subsection{Lua module}
%
%    \begin{macrocode}
%<*lua>
%    \end{macrocode}
%
%    \subsubsection{Module identification}
%
%    \begin{macrocode}
module('luatexbase', package.seeall)
local err, warning, info = luatexbase.provides_module({
    name          = "luatexbase-mcb",
    version       = 0.2,
    date          = "2010/05/12",
    description   = "register several functions in a callback",
    author        = "Hans Hagen, Elie Roux and Manuel Pegourie-Gonnard",
    copyright     = "Hans Hagen, Elie Roux and Manuel Pegourie-Gonnard",
    license       = "CC0",
})
%    \end{macrocode}
%
%    \subsubsection{Housekeeping}
%
%    The main table: keys are callback names, and values are the associated
%    lists of functions. More precisely, the entries in the list are tables
%    holding the actual function as |func| and the identifying description as
%    |description|. Only callbacks with a non-empty list of functions have an
%    entry in this list.
%
%    \begin{macrocode}
local callbacklist = callbacklist or { }
%    \end{macrocode}
%
%    Numerical codes for callback types, and name to value association (the
%    table keys are strings, the values are numbers).
%
%    \begin{macrocode}
local list, data, first, simple = 1, 2, 3, 4
local types = {
    list   = list,
    data   = data,
    first  = first,
    simple = simple,
}
%    \end{macrocode}
%
%    Now, list all predefined callbacks with their current type, based on the
%    LuaTeX manual version 0.60.2.
%
%    \begin{macrocode}
local callbacktypes = callbacktypes or {
%    \end{macrocode}
%
%    Section 4.1.1: file discovery callbacks.
%
%    \begin{macrocode}
    find_read_file     = first,
    find_write_file    = first,
    find_font_file     = data,
    find_output_file   = data,
    find_format_file   = data,
    find_vf_file       = data,
    find_ocp_file      = data,
    find_map_file      = data,
    find_enc_file      = data,
    find_sfd_file      = data,
    find_pk_file       = data,
    find_data_file     = data,
    find_opentype_file = data,
    find_truetype_file = data,
    find_type1_file    = data,
    find_image_file    = data,
%    \end{macrocode}
%
%    Section 4.1.2: file reading callbacks.
%
%    \begin{macrocode}
    open_read_file     = first,
    read_font_file     = first,
    read_vf_file       = first,
    read_ocp_file      = first,
    read_map_file      = first,
    read_enc_file      = first,
    read_sfd_file      = first,
    read_pk_file       = first,
    read_data_file     = first,
    read_truetype_file = first,
    read_type1_file    = first,
    read_opentype_file = first,
%    \end{macrocode}
%
%    Section 4.1.3: data processing callbacks.
%
%    \begin{macrocode}
    process_input_buffer  = data,
    process_output_buffer = data,
    token_filter          = first,
%    \end{macrocode}
%
%    Section 4.1.4: node list processiong callbacks.
%
%    \begin{macrocode}
    buildpage_filter      = simple,
    pre_linebreak_filter  = list,
    linebreak_filter      = list,
    post_linebreak_filter = list,
    hpack_filter          = list,
    vpack_filter          = list,
    pre_output_filter     = list,
    hyphenate             = simple,
    ligaturing            = simple,
    kerning               = simple,
    mlist_to_hlist        = list,
%    \end{macrocode}
%
%    Section 4.1.5: information reporting callbacks.
%
%    \begin{macrocode}
    start_run         = simple,
    stop_run          = simple,
    start_page_number = simple,
    stop_page_number  = simple,
    show_error_hook   = simple,
%    \end{macrocode}
%
%    Section 4.1.6: font-related callbacks.
%
%    \begin{macrocode}
    define_font = first,
}
%    \end{macrocode}
%
%    All user-defined callbacks have a default function. The following table's
%    keys are the names of the user-defined callback, the associated value is
%    the default functon for this callback. This table is also used to
%    identify the user-defined callbacks.
%
%    \begin{macrocode}
local lua_callbacks_defaults = { }
%    \end{macrocode}
%
%    Overwrite |callback.register|, but save it first. Also define a wrapper
%    that automatically raise an error when something goes wrong.
%
%    \begin{macrocode}
local original_register = original_register or callback.register
callback.register = function ()
  err("function callback.register has been trapped,\n"
  .."please use luatexbase.add_to_callback instead.")
end
local function register_callback(...)
    return assert(original_register(...))
end
%    \end{macrocode}
%
%    \subsubsection{Handlers}
%
%    Normal (as opposed to user-defined) callbacks have handlers depending on
%    their type. The handler function is registered into the callback when the
%    first function is added to this callback's list. Then, when the callback
%    is called, then handler takes care of running all functions in the list.
%    When the last function is removed from the callback's list, the handler
%    is unregistered.
%
%    More precisely, the functions below are used to generate a specialized
%    function (closure) for a given callback, which is the actual handler.
%
%    Handler for |list| callbacks.
%
%    \begin{macrocode}
local function listhandler (name)
    return function(head,...)
        local ret
        local alltrue = true
        for _, f in ipairs(callbacklist[name]) do
            ret = f.func(head, ...)
            if ret == false then
                warn("function '%s' returned false\nin callback '%s'",
                    f.description, name)
                break
            end
            if ret ~= true then
                alltrue = false
                head = ret
            end
        end
        return alltrue and true or head
    end
end
%    \end{macrocode}
%
%    Handler for |data| callbacks.
%
%    \begin{macrocode}
local function datahandler (name)
    return function(data, ...)
        for _, f in ipairs(callbacklist[name]) do
            data = f.func(data, ...)
        end
        return data
    end
end
%    \end{macrocode}
%
%    Handler for |first| callbacks. We can assume |callbacklist[name]| is not
%    empty: otherwise, the function wouldn't be registered in the callback any
%    more.
%
%    \begin{macrocode}
local function firsthandler (name)
    return function(...)
        return callbacklist[name][1].func(...)
    end
end
%    \end{macrocode}
%
%    Handler for |simple| callbacks.
%
%    \begin{macrocode}
local function simplehandler (name)
    return function(...)
        for _, f in ipairs(callbacklist[name]) do
            f.func(...)
        end
    end
end
%    \end{macrocode}
%
%    Finally, keep a handlers table for indexed access.
%
%    \begin{macrocode}
local handlers = {
  [list]   = listhandler,
  [data]   = datahandler,
  [first]  = firsthandler,
  [simple] = simplehandler,
}
%    \end{macrocode}
%
%    \subsubsection{Public functions for functions management}
%
%    Add a function to a callback. First check arguments.
%
%    \begin{macrocode}
function add_to_callback (name,func,description,priority)
    if type(func) ~= "function" then
        return err("unable to add function:\nno proper function passed")
    end
    if not name or name == "" then
        err("unable to add function:\nno proper callback name passed")
        return
    elseif not callbacktypes[name] then
        err("unable to add function:\n'%s' is not a valid callback", name)
        return
    end
    if not description or description == "" then
        err("unable to add function to '%s':\nno proper description passed",
          name)
        return
    end
    if priority_in_callback(name, description) then
        err("function '%s' already registered\nin callback '%s'",
          description, name)
        return
    end
%    \end{macrocode}
%
%    Then test if this callback is already in use. If not, initialise its list
%    and register the proper handler.
%
%    \begin{macrocode}
    local l = callbacklist[name]
    if not l then
        l = {}
        callbacklist[name] = l
        if not lua_callbacks_defaults[name] then
            register_callback(name, handlers[callbacktypes[name]](name))
        end
    end
%    \end{macrocode}
%
%    Actually register the function.
%
%    \begin{macrocode}
    local f = {
        func = func,
        description = description,
    }
    priority = tonumber(priority)
    if not priority or priority > #l then
        priority = #l+1
    elseif priority < 1 then
        priority = 1
    end
    table.insert(l,priority,f)
%    \end{macrocode}
%
%    Keep user informed.
%
%    \begin{macrocode}
    if callbacktypes[name] == first and #l ~= 1 then
        warning("several functions in '%s',\n"
        .."only one will be active.", name)
    end
    info("inserting '%s'\nat position %s in '%s'",
      description, priority, name)
end
%    \end{macrocode}
%
%    Remove a function from a callback. First check arguments.
%
%    \begin{macrocode}
function remove_from_callback (name, description)
    if not name or name == "" then
        err("unable to remove function:\nno proper callback name passed")
        return
    elseif not callbacktypes[name] then
        err("unable to remove function:\n'%s' is not a valid callback", name)
        return
    end
    if not description or description == "" then
        err(
          "unable to remove function from '%s':\nno proper description passed",
          name)
        return
    end
    local l = callbacklist[name]
    if not l then
        err("no callback list for '%s'",name)
        return
    end
%    \end{macrocode}
%
%    Then loop over the callback's function list until we find a matching
%    entry. Remove it and check if the list gets empty: if so, unregister the
%    callback handler unless the callback is user-defined.
%
%    \begin{macrocode}
    local index = false
    for k,v in ipairs(l) do
        if v.description == description then
            index = k
            break
        end
    end
    if not index then
        err("unable to remove '%s'\nfrom '%s'", description, name)
        return
    end
    table.remove(l, index)
    info("removing '%s'\nfrom '%s'", description, name)
    if table.maxn(l) == 0 then
        callbacklist[name] = nil
        if not lua_callbacks_defaults[name] then
            register_callback(name, nil)
        end
    end
    return
end
%    \end{macrocode}
%
%    Remove all the functions registered in a callback. Unregisters the
%    callback handler unless the callback is user-defined.
%
%    \begin{macrocode}
function reset_callback (name, make_false)
    if not name or name == "" then
        err("unable to reset:\nno proper callback name passed")
        return
    elseif not callbacktypes[name] then
        err("unable to reset '%s':\nis not a valid callback", name)
        return
    end
    info("resetting callback '%s'", name)
    callbacklist[name] = nil
    if not lua_callbacks_defaults[name] then
        if make_false == true then
            info("setting '%s' to false", name)
            register_callback(name, false)
        else
            register_callback(name, nil)
        end
    end
end
%    \end{macrocode}
%
%    Get a function's priority in a callback list, or false if the function is
%    not in the list.
%
%    \begin{macrocode}
function priority_in_callback (name, description)
    if not name or name == ""
            or not callbacktypes[name]
            or not description then
        return false
    end
    local l = callbacklist[name]
    if not l then return false end
    for p, f in pairs(l) do
        if f.description == description then
            return p
        end
    end
    return false
end
%    \end{macrocode}
%
%    \subsubsection{Public functions for user-defined callbacks}
%
%    This first function creates a new callback. The signature is
%    \texttt{create(name, ctype, default)} where \texttt{name} is the name of
%    the new callback to create, \texttt{ctype} is the type of callback, and
%    \texttt{default} is the default function to call if no function is
%    registered in this callback.
%
%    The created callback will behave the same way Lua\TeX\ callbacks do, you
%    can add and remove functions in it. The difference is that the callback
%    is not automatically called, the package developer creating a new
%    callback must also call it, see next function.
%
%    \begin{macrocode}
function create_callback(name, ctype, default)
    if not name then
        err("unable to call callback:\nno proper name passed", name)
        return nil
    end
    if not ctype or not default then
        err("unable to create callback '%s':\n"
        .."callbacktype or default function not specified", name)
        return nil
    end
    if callbacktypes[name] then
        err("unable to create callback '%s':\ncallback already exists", name)
        return nil
    end
    ctype = types[ctype]
    if not ctype then
        err("unable to create callback '%s':\ntype '%s' undefined", name, ctype)
        return nil
    end
    info("creating '%s' type %s", name, ctype)
    lua_callbacks_defaults[name] = default
    callbacktypes[name] = ctype
end
%    \end{macrocode}
%
%    This function calls a callback. It can only call a callback created by
%    the \texttt{create} function.
%
%    \begin{macrocode}
function call_callback(name, ...)
    if not name then
        err("unable to call callback:\nno proper name passed", name)
        return nil
    end
    if not lua_callbacks_defaults[name] then
        err("unable to call lua callback '%s':\nunknown callback", name)
        return nil
    end
    local l = callbacklist[name]
    local f
    if not l then
        f = lua_callbacks_defaults[name]
    else
        f = handlers[callbacktypes[name]](name)
        if not f then
            err("unknown callback type")
            return
        end
    end
    return f(...)
end
%    \end{macrocode}
%
%    That's all folks!
%
%    \begin{macrocode}
%</lua>
%    \end{macrocode}
%
%    \section{Test files}
%
%    A few basic tests for Plain and LaTeX. Use a separate Lua file for
%    convenience, since this package works on the Lua side of the force.
%
%    \begin{macrocode}
%<*testlua>
local msg = texio.write_nl
%    \end{macrocode}
%
%    Test the management functions with a predefined callback.
%
%    \begin{macrocode}
local function sample(head,...)
    return head, true
end
local prio = luatexbase.priority_in_callback
msg("\n*********\n* Testing management functions\n*********")
luatexbase.add_to_callback("hpack_filter", sample, "sample one", 1)
luatexbase.add_to_callback("hpack_filter", sample, "sample two", 2)
luatexbase.add_to_callback("hpack_filter", sample, "sample three", 1)
assert(prio("hpack_filter", "sample three"))
luatexbase.remove_from_callback("hpack_filter", "sample three")
assert(not prio("hpack_filter", "sample three"))
luatexbase.reset_callback("hpack_filter")
assert(not prio("hpack_filter", "sample one"))
%    \end{macrocode}
%
%    Create a callback, and check that the managment functions work with this
%    callback too.
%
%    \begin{macrocode}
local function data_one(s)
  texio.write_nl("I'm data 1 whith argument: "..s)
  return s
end
local function data_two(s)
  texio.write_nl("I'm data 2 whith argument: "..s)
  return s
end
local function data_three(s)
  texio.write_nl("I'm data 3 whith argument: "..s)
  return s
end
msg("\n*********\n* Testing user-defined callbacks\n*********")
msg("* create one")
luatexbase.create_callback("fooback", "data", data_one)
msg("* call it")
luatexbase.call_callback("fooback", "default")
msg("* add two functions")
luatexbase.add_to_callback("fooback", data_two, "function two", 2)
luatexbase.add_to_callback("fooback", data_three, "function three", 1)
msg("* call")
luatexbase.call_callback("fooback", "all")
msg("* rm one function")
luatexbase.remove_from_callback("fooback", "function three")
msg("* call")
luatexbase.call_callback("fooback", "all but three")
msg("* reset")
luatexbase.reset_callback("fooback")
msg("* call")
luatexbase.call_callback("fooback", "default")
%    \end{macrocode}
%
%    Now, we want to make each handler run at least once. So, define dummy
%    functions and register them in various callbacks. We will make sure the
%    callbacks are executed on the \tex end. Also, we want to check that
%    everything works when we unload the functions either one by one, or using
%    reset.
%
%    A |list| callback.
%
%    \begin{macrocode}
function add_hpack_filter()
    luatexbase.add_to_callback('hpack_filter', function(head, ...)
            texio.write_nl("I'm a dummy hpack_filter")
            return head
        end,
        'dummy hpack filter')
    luatexbase.add_to_callback('hpack_filter', function(head, ...)
            texio.write_nl("I'm an optimized dummy hpack_filter")
            return true
        end,
        'optimized dummy hpack filter')
end
function rm_one_hpack_filter()
    luatexbase.remove_from_callback('hpack_filter', 'dummy hpack filter')
end
function rm_two_hpack_filter()
    luatexbase.remove_from_callback('hpack_filter',
        'optimized dummy hpack filter')
end
%    \end{macrocode}
%
%    A |simple| callback.
%
%    \begin{macrocode}
function add_hyphenate()
    luatexbase.add_to_callback('hyphenate', function(head, tail)
            texio.write_nl("I'm a dummy hyphenate")
        end,
        'dummy hyphenate')
    luatexbase.add_to_callback('hyphenate', function(head, tail)
            texio.write_nl("I'm an other dummy hyphenate")
        end,
        'other dummy hyphenate')
end
function rm_one_hyphenate()
    luatexbase.remove_from_callback('hyphenate', 'dummy hyphenate')
end
function rm_two_hyphenate()
    luatexbase.remove_from_callback('hyphenate', 'other dummy hyphenate')
end
%    \end{macrocode}
%
%    A |first| callback.
%
%    \begin{macrocode}
function add_find_write_file()
    luatexbase.add_to_callback('find_write_file', function(id, name)
            texio.write_nl("I'm a dummy find_write_file")
            return "dummy-"..name
        end,
        'dummy find_write_file')
    luatexbase.add_to_callback('find_write_file', function(id, name)
            texio.write_nl("I'm an other dummy find_write_file")
            return "dummy-other-"..name
        end,
        'other dummy find_write_file')
end
function rm_one_find_write_file()
    luatexbase.remove_from_callback('find_write_file',
        'dummy find_write_file')
end
function rm_two_find_write_file()
    luatexbase.remove_from_callback('find_write_file',
        'other dummy find_write_file')
end
%    \end{macrocode}
%
%    A |data| callback.
%
%    \begin{macrocode}
function add_process_input_buffer()
    luatexbase.add_to_callback('process_input_buffer', function(buffer)
            return buffer.."\\msg{dummy}"
        end,
        'dummy process_input_buffer')
    luatexbase.add_to_callback('process_input_buffer', function(buffer)
            return buffer.."\\msg{otherdummy}"
        end,
        'other dummy process_input_buffer')
end
function rm_one_process_input_buffer()
    luatexbase.remove_from_callback('process_input_buffer',
        'dummy process_input_buffer')
end
function rm_two_process_input_buffer()
    luatexbase.remove_from_callback('process_input_buffer',
        'other dummy process_input_buffer')
end
%    \end{macrocode}
%
%    \begin{macrocode}
%</testlua>
%    \end{macrocode}
%
%    \begin{macrocode}
%<testplain>\input luatexbase-mcb.sty
%<testlatex>\RequirePackage{luatexbase-mcb}
%<*testplain,testlatex>
\catcode 64 11
\def\msg{\immediate\write16}
\msg{===== BEGIN =====}
%    \end{macrocode}
%
%    Loading the lua files tests that the management functions can be called
%    without raising errors.
%
%    \begin{macrocode}
\luatexbase@directlua{dofile('test-mcb.lua')}
%    \end{macrocode}
%
%    We now want to load and unload stuff from the various callbacks have them
%    called to test the handlers. Here is a helper macro for that.
%
%    \begin{macrocode}
\def\test#1#2{%
  \msg{^^J*********^^J* Testing #1 (type #2)^^J*********}
  \msg{* Add two functions}
  \luatexbase@directlua{add_#1()}
  \csname test_#1\endcsname
  \msg{* Remove one}
  \luatexbase@directlua{rm_one_#1()}
  \csname test_#1\endcsname
  \msg{* Remove the second}
  \luatexbase@directlua{rm_two_#1()}
  \csname test_#1\endcsname
  \msg{* Add two functions again}
  \luatexbase@directlua{add_#1()}
  \csname test_#1\endcsname
  \msg{* Remove all functions}
  \luatexbase@directlua{luatexbase.reset_callback("#1")}
  \csname test_#1\endcsname
}
%    \end{macrocode}
%
%    For each callback, we need a specific macro that triggers it. For the
%    hyphenate test, we need to untrap |\everypar| first, in the \latex case.
%
%    \begin{macrocode}
\catcode`\_ 11
%<testlatex>\everypar{}
\def\test_hpack_filter{\setbox0=\hbox{bla}}
\def\test_hyphenate{\showhyphens{hyphenation}}
\def\test_find_write_file{\immediate\openout15 test-mcb-out.log}
\def\test_process_input_buffer{\input test-mcb-aux.tex}
%    \end{macrocode}
%
%    Now actually test them
%
%    \begin{macrocode}
\test{hpack_filter}{list}
\test{hyphenate}{simple}
\test{find_write_file}{first}
\test{process_input_buffer}{data}
%    \end{macrocode}
%
%    Done.
%
%    \begin{macrocode}
\msg{=====  END =====}
%</testplain,testlatex>
%<testplain>\bye
%<testlatex>\stop
%    \end{macrocode}
%
% \Finale
\endinput
