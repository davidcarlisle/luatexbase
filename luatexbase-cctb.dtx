% \iffalse meta-comment
%
% Written in 2009, 2010 by Manuel Pégourié-Gonnard and Élie Roux.
%     <mpg@elzevir.fr>
%     <elie.roux@telecom-bretagne.eu>
%
% This work is under the CC0 license.
%
% This work consists of the main source file luatexbase-cctb.dtx
% and the derived files
%    luatexbase-cctb.sty luatexbase.cctb.lua
%    test-cctb-plain.tex test-cctb-latex.tex
%
% Unpacking:
%    tex luatexbase-cctb.dtx
% Documentation:
%    pdflatex luatexbase-cctb.dtx
%
%<*ignore>
\begingroup
  \def\x{LaTeX2e}%
\expandafter\endgroup
\ifcase 0\ifx\install y1\fi\expandafter
         \ifx\csname processbatchFile\endcsname\relax\else1\fi
         \ifx\fmtname\x\else 1\fi\relax
\else\csname fi\endcsname
%</ignore>
%<*install>
\input docstrip.tex

\keepsilent
\askforoverwritefalse

\let\MetaPrefix\relax

\preamble
This is a generated file.

Written in 2009, 2010 by Manuel Pégourié-Gonnard and Élie Roux.
    <mpg@elzevir.fr>
    <elie.roux@telecom-bretagne.eu>

This work is under the CC0 license.

This work consists of the main source file luatexbase-cctb.dtx
and the derived files
   luatexbase-cctb.sty luatexbase.cctb.lua
   test-cctb-plain.tex test-cctb-latex.tex

\endpreamble

\let\MetaPrefix\DoubleperCent

\generate{%
  \usedir{tex/luatex/luatexbase}%
  \file{luatexbase-cctb.sty}{\from{luatexbase-cctb.dtx}{texpackage}}%
}

\generate{%
  \usedir{doc/luatex/luatexbase}%
  \file{test-cctb-plain.tex}{\from{luatexbase-cctb.dtx}{testplain}}%
  \file{test-cctb-latex.tex}{\from{luatexbase-cctb.dtx}{testlatex}}%
}

\def\MetaPrefix{-- }

\def\luapostamble{%
  \MetaPrefix^^J%
  \MetaPrefix\space End of File `\outFileName'.%
}

\def\currentpostamble{\luapostamble}%

\generate{%
  \usedir{tex/luatex/luatexbase}%
  \file{luatexbase.cctb.lua}{\from{luatexbase-cctb.dtx}{luamodule}}%
}

\obeyspaces
\Msg{************************************************************************}
\Msg{*}
\Msg{* To finish the installation you have to move the following}
\Msg{* files into a directory searched by TeX:}
\Msg{*}
\Msg{*     luatexbase-cctb.sty cctb.lua ...}
\Msg{*}
\Msg{* Happy TeXing!}
\Msg{*}
\Msg{************************************************************************}

\endbatchfile
%</install>
%<*ignore>
\fi
%</ignore>
%<*driver>
\documentclass{ltxdoc}
% preamble for dtx documentations of the luatexbase package/bundle.

\usepackage[T1]{fontenc}
\usepackage{lmodern}
\usepackage{geometry}
\usepackage{xspace}
\usepackage[english]{babel}
\usepackage[colorlinks]{hyperref}
\usepackage{bookmark}

% logos
\makeatletter
\newcommand\eTeX{$\m@th\varepsilon$-\TeX}
\newcommand\LuaTeX{Lua\TeX}
\renewcommand\PlainTeX{Plain\thinspace\TeX}
\newcommand\TeXe{\TeX\thinspace82}
\newcommand\TeXLive{\TeX\thinspace Live}
\makeatother

% logos for the lazy me (mpg)
\newcommand\tex{\TeX\xspace}
\newcommand\latex{\LaTeX\xspace}
\newcommand\etex{\eTeX\xspace}
\newcommand\luatex{\LuaTeX\xspace}
\newcommand\texe{\TeXe\xspace}
\newcommand\texlive{\TeXLive\xspace}
\newcommand\plaintex{\PlainTeX\xspace}

% special elements of text
\newcommand\file{\nolinkurl}
\newcommand\email[1]{\href{mailto:#1}{#1}}
\newcommand\pk{\textsf}
\newcommand\cmdname{\texttt}

% for hyperref
\pdfstringdefDisableCommands{%
  \def\cs#1{\@backslashchar #1}%
  }

% easy verbatim
\MakeShortVerb\|

\begin{document}
  \DocInput{luatexbase-cctb.dtx}%
\end{document}
%</driver>
% \fi
%
% \CheckSum{0}
%
% \CharacterTable
%  {Upper-case    \A\B\C\D\E\F\G\H\I\J\K\L\M\N\O\P\Q\R\S\T\U\V\W\X\Y\Z
%   Lower-case    \a\b\c\d\e\f\g\h\i\j\k\l\m\n\o\p\q\r\s\t\u\v\w\x\y\z
%   Digits        \0\1\2\3\4\5\6\7\8\9
%   Exclamation   \!     Double quote  \"     Hash (number) \#
%   Dollar        \$     Percent       \%     Ampersand     \&
%   Acute accent  \'     Left paren    \(     Right paren   \)
%   Asterisk      \*     Plus          \+     Comma         \,
%   Minus         \-     Point         \.     Solidus       \/
%   Colon         \:     Semicolon     \;     Less than     \<
%   Equals        \=     Greater than  \>     Question mark \?
%   Commercial at \@     Left bracket  \[     Backslash     \\
%   Right bracket \]     Circumflex    \^     Underscore    \_
%   Grave accent  \`     Left brace    \{     Vertical bar  \|
%   Right brace   \}     Tilde         \~}
%
% \title{The \pk{luatexbase-cctb} package}
% \date{v0.1 2010-03-26}
% \author{%
%  Manuel P\'egouri\'e-Gonnard \\ \texttt{mpg@elzevir.fr} \and
%   \'Elie Roux \\ \texttt{elie.roux@telecom-bretagne.eu}}
%
% \maketitle
%
% \begin{abstract}
% \end{abstract}
%
% \section{Documentation}
%
%    \section{Implementation}
%
%    \subsection{\tex package}
%
%    \begin{macrocode}
%<*texpackage>
%    \end{macrocode}
%
%    \subsubsection{Preliminaries}
%
%    Reload protection, especially for \plaintex.
%
%    \begin{macrocode}
                \csname lltxb@cctb@loaded\endcsname
\expandafter\let\csname lltxb@cctb@loaded\endcsname\endinput
%    \end{macrocode}
%
%    Package declaration.
%
%    \begin{macrocode}
\begingroup
  \expandafter\ifx\csname ProvidesPackage\endcsname\relax
    \def\x#1[#2]{\immediate\write16{Package: #1 #2}}
  \else
    \let\x\ProvidesPackage
  \fi
\expandafter\endgroup
\x{luatexbase-cctb}[2010/03/26 v0.1  Catcodetable allocation for LuaTeX  (mpg)]
%    \end{macrocode}
%
%    Make sure \luatex is used.
%
%    \begin{macrocode}
\begingroup\expandafter\expandafter\expandafter\endgroup
\expandafter\ifx\csname RequirePackage\endcsname\relax
  \input ifluatex.sty
\else
  \RequirePackage{ifluatex}
\fi
\ifluatex\else
  \begingroup
    \expandafter\ifx\csname PackageWarningNoLine\endcsname\relax
      \def\x#1#2{\begingroup\newlinechar10
        \immediate\write16{Package #1 warning: #2}\endgroup}
    \else
      \let\x\PackageWarningNoLine
    \fi
  \expandafter\endgroup
  \x{luatexbase-cctb}{LuaTeX is required for this package.^^J
    Aborting package loading.}
  \expandafter\endinput
\fi
%    \end{macrocode}
%
%    Make sure the catcode of @ is correct, especially for \plaintex.
%
%    \begin{macrocode}
\expandafter\edef\csname lltxb@cctb@AtEnd\endcsname{%
  \catcode64 \the\catcode64\relax}
\catcode64 11
%    \end{macrocode}
%
%    \subsubsection{Main content}
%
%    Load the supporting Lua module.
%
%    \begin{macrocode}
\directlua{dofile(kpse.find_file('luatexbase.cctb.lua', 'lua'))}
%    \end{macrocode}
%
%    The allocation macro. Allocate tables starting with 1, since table 0 is
%    reserved for IniTeX catcodes by LuaTeX.
%
%    \begin{macrocode}
\newcount\lltxb@catcodetable@alloc
\lltxb@catcodetable@alloc\z@
\def\newluatexcatcodetable#1{%
  \ifnum\lltxb@catcodetable@alloc<65535\relax
    \global\advance\lltxb@catcodetable@alloc\@ne
    \allocationnumber\lltxb@catcodetable@alloc
    \global\chardef#1\allocationnumber
    \luatexinitcatcodetable\allocationnumber
    \begingroup\escapechar\m@ne \expandafter\endgroup
    \directlua{luatextra.catcodetabledef_from_tex(
      '\luatexluaescapestring{\string#1}', '\number\allocationnumber')}%
    \wlog{\string#1=\string\luatexcatcodetable\the\allocationnumber}%
  \else
    \errmessage{No room for a new \string\luatexcatcodetable}%
  \fi}
%    \end{macrocode}
%
%    A small patch to get two new counters in Plain too.
%
%    \begin{macrocode}
\expandafter\ifx\csname @tempcnta\endcsname\relax
  \csname newcount\endcsname\@tempcnta
\fi
\expandafter\ifx\csname @tempcntb\endcsname\relax
  \csname newcount\endcsname\@tempcntb
\fi
%    \end{macrocode}
%
%    A macro that sets the catcode of a range of characters. The first
%    parameter is the character number of the first character of the range,
%    the second parameter is one for the last character, and the third
%    parameter is the catcode we want them to have.
%
%    \begin{macrocode}
\def\luatexsetcatcoderange#1#2#3{%
  \edef\luaSCR@temp{%
    \noexpand\@tempcnta=\the\@tempcnta
    \noexpand\@tempcntb=\the\@tempcntb
    \noexpand\count@=\the\count@
    \relax
  }%
  \@tempcnta=#1\relax
  \@tempcntb=#2\relax
  \count@=#3\relax
  \loop\unless\ifnum\@tempcnta>\@tempcntb
    \catcode\@tempcnta=\count@
    \advance\@tempcnta by 1\relax
  \repeat
  \luaSCR@temp}
%    \end{macrocode}
%
%    Finally we create several catcodetables that may be used by the user.
%    These are:
%
%    \begin{itemize}
%    \item \texttt{\string\CatcodeTableIniTeX}: the base \TeX\ catcodes
%    \item \texttt{\string\CatcodeTableString}: almost all characters have
%    catcode 12
%    \item \texttt{\string\CatcodeTableOther}: all characters have catcode 12
%    (even space)
%    \item \texttt{\string\CatcodeTableLaTeX}: the \LaTeX\ classical catcodes
%    \item \texttt{\string\CatcodeTableLaTeXAtLetter}: the \LaTeX\ classical
%    catcodes and |@| letter
%    \item \texttt{\string\CatcodeTableExpl}: the expl3 catcodes
%    \end{itemize}
%
%    \begin{macrocode}
\def\setluatexcatcodetable#1#2{%
  \begingroup
  #2%
  \luatexsavecatcodetable#1%
  \endgroup}
%    \end{macrocode}
%
%    \begin{macrocode}
\newluatexcatcodetable\CatcodeTableIniTeX
%    \end{macrocode}
%
%    \begin{macrocode}
\newluatexcatcodetable\CatcodeTableString
\setluatexcatcodetable\CatcodeTableString{%
  \luatexcatcodetable\CatcodeTableIniTeX
  \catcode0   12 % nul
  \catcode13  12 % carriage return
  \catcode37  12 % percent
  \luatexsetcatcoderange{65}{90}{12}% A-Z
  \luatexsetcatcoderange{97}{122}{12}% a-z
  \catcode92  12 % backslash
  \catcode127 12 }
%    \end{macrocode}
%
%    \begin{macrocode}
\newluatexcatcodetable\CatcodeTableOther
\setluatexcatcodetable\CatcodeTableOther{%
  \luatexcatcodetable\CatcodeTableString
  \catcode32 12 }
%    \end{macrocode}
%
%    \begin{macrocode}
\newluatexcatcodetable\CatcodeTableLaTeX
\setluatexcatcodetable\CatcodeTableLaTeX{%
  \luatexcatcodetable\CatcodeTableIniTeX
  \luatexsetcatcoderange{0}{31}{15}%
  \catcode9   10 % tab
  \catcode12  13 % form feed
  \catcode13  5  % carriage return
  \catcode35  6  % hash
  \catcode36  3  % dollar
  \catcode38  4  % ampersand
  \catcode94  7  % circumflex
  \catcode95  8  % underscore
  \catcode123 1  % brace left
  \catcode125 2  % brace right
  \catcode126 13 % tilde
  \catcode127 15 }
%    \end{macrocode}
%
%    \begin{macrocode}
\newluatexcatcodetable\CatcodeTableLaTeXAtLetter
\setluatexcatcodetable\CatcodeTableLaTeXAtLetter{%
  \luatexcatcodetable\CatcodeTableLaTeX
  \catcode64=11 }
%    \end{macrocode}
%
%    \begin{macrocode}
\newluatexcatcodetable\CatcodeTableExpl
\setluatexcatcodetable\CatcodeTableExpl{%
  \luatexcatcodetable\CatcodeTableLaTeX
  \catcode126 10 % tilde is a space char
  \catcode32  9  % space is ignored
  \catcode9   9  % tab also ignored
  \catcode95  11 % underscore letter
  \catcode58  11 % colon letter
  }
%    \end{macrocode}
%
%    \begin{macrocode}
\directlua{luatextra.catcodetable_do_shortcuts()}
%</texpackage>
%    \end{macrocode}
%
%    \subsection{Lua module}
%
%    \begin{macrocode}
%<*luamodule>
module('luatextra', package.seeall)
%    \end{macrocode}
%
%    In the same way, the table \texttt{tex.catcodetablenumber} contains the
%    numbers of the catcodetables registered with
%    \texttt{\string\newluacatcodetable}.
%
%    \begin{macrocode}
luatextra.catcodetables = {}
tex.catcodetablenumber = luatextra.catcodetables
function luatextra.catcodetabledef_from_tex(name, number)
    luatextra.catcodetables[name] = tonumber(number)
end
%    \end{macrocode}
%
%    With this function we create some shortcuts for a better readability in
%    lua code. This makes |tex.catcodetablenumber.latex| equivalent to
%    |tex.catcodetablenumber['CatcodeTableLaTeX']|.
%
%    \begin{macrocode}
function luatextra.catcodetable_do_shortcuts()
    local cat = luatextra.catcodetables
    cat['latex']                = cat.CatcodeTableLaTeX
    cat['latex-package']        = cat.CatcodeTableLaTeXAtLetter
    cat['latex-atletter']       = cat.CatcodeTableLaTeXAtLetter
    cat['ini']                  = cat.CatcodeTableIniTeX
    cat['expl3']                = cat.CatcodeTableExpl
    cat['expl']                 = cat.CatcodeTableExpl
    cat['string']               = cat.CatcodeTableString
    cat['other']                = cat.CatcodeTableOther
end
%    \end{macrocode}
%
%    \begin{macrocode}
%</luamodule>
%    \end{macrocode}
%
%    \section{Test files}
%
%    The tests done are very basic: we just make sure that the package loads
%    correctly and the macros don't generate any error, under both LaTeX en
%    Plain TeX.
%
%    \begin{macrocode}
%<testplain>\input luatexbase-cctb.sty
%<testlatex>\RequirePackage{luatexbase-cctb}
%<*testplain,testlatex>
\newluatexcatcodetable\testcctb
\directlua{assert(luatextra.catcodetables.testcctb)}
%    \end{macrocode}
%
%    Also check that the catcodetable's number is remembered well,
%    independently of the current value of |\escapechar|.

%    \begin{macrocode}
\begingroup
\escapechar64
\newluatexcatcodetable\anothercctb
\endgroup
\directlua{assert(luatextra.catcodetables.anothercctb)}
%    \end{macrocode}
%
%    Now, play a little bit with predefined tables.
%
%    \begin{macrocode}
\luatexcatcodetable\CatcodeTableLaTeXAtLetter
\ifnum\catcode64=11 \else \ERROR \fi
\luatexcatcodetable\CatcodeTableLaTeX
\ifnum\catcode64=12 \else \ERROR \fi
%<testlatex>\documentclass{minimal}
\directlua{%
  tex.sprint('\string\\hbox{')
  tex.sprint(luatextra.catcodetables.string, "\string\\undef # _^&")
  tex.sprint('}')
  }
%    \end{macrocode}
%
%    \begin{macrocode}
%</testplain,testlatex>
%<testplain>\bye
%<testlatex>\stop
%    \end{macrocode}
%
% \Finale
\endinput
