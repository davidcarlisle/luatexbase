% \iffalse meta-comment
%
% Written in 2009, 2010 by Manuel Pégourié-Gonnard and Élie Roux.
%     <mpg@elzevir.fr>
%     <elie.roux@telecom-bretagne.eu>
%
% This work is under the CC0 license.
%
% This work consists of the main source file luatexbase-modutils.dtx
% and the derived files
%    luatexbase-modutils.sty modutils.lua
%    test-modutils-plain.tex test-modutils-latex.tex test-modutils.lua
%
% Unpacking:
%    tex luatexbase-modutils.dtx
% Documentation:
%    pdflatex luatexbase-modutils.dtx
%
%<*ignore>
\begingroup
  \def\x{LaTeX2e}%
\expandafter\endgroup
\ifcase 0\ifx\install y1\fi\expandafter
         \ifx\csname processbatchFile\endcsname\relax\else1\fi
         \ifx\fmtname\x\else 1\fi\relax
\else\csname fi\endcsname
%</ignore>
%<*install>
\input docstrip.tex

\keepsilent
\askforoverwritefalse

\let\MetaPrefix\relax

\preamble

Written in 2009, 2010 by Manuel Pegourie-Gonnard and Elie Roux.

This work is under the CC0 license.
See source file '\inFileName' for details.

\endpreamble

\let\MetaPrefix\DoubleperCent

\generate{%
  \usedir{tex/luatex/luatexbase}%
  \file{luatexbase-modutils.sty}{\from{luatexbase-modutils.dtx}{texpackage}}%
}

\generate{%
  \usedir{doc/luatex/luatexbase}%
  \file{test-modutils-plain.tex}{\from{luatexbase-modutils.dtx}{testplain}}%
  \file{test-modutils-latex.tex}{\from{luatexbase-modutils.dtx}{testlatex}}%
}

\def\MetaPrefix{-- }

\def\luapostamble{%
  \MetaPrefix^^J%
  \MetaPrefix\space End of File `\outFileName'.%
}

\def\currentpostamble{\luapostamble}%

\generate{%
  \usedir{tex/luatex/luatexbase}%
  \file{modutils.lua}{\from{luatexbase-modutils.dtx}{luamodule}}%
  \usedir{doc/luatex/luatexbase}%
  \file{test-modutils.lua}{\from{luatexbase-modutils.dtx}{testdummy}}%
}

\obeyspaces
\Msg{************************************************************************}
\Msg{*}
\Msg{* To finish the installation you have to move the following}
\Msg{* files into a directory searched by TeX:}
\Msg{*}
\Msg{*     luatexbase-modutils.sty modutils.lua}
\Msg{*}
\Msg{* Happy TeXing!}
\Msg{*}
\Msg{************************************************************************}

\endbatchfile
%</install>
%<*ignore>
\fi
%</ignore>
%<*driver>
\documentclass{ltxdoc}
% preamble for dtx documentations of the luatexbase package/bundle.

\usepackage[T1]{fontenc}
\usepackage{lmodern}
\usepackage{geometry}
\usepackage{xspace}
\usepackage[english]{babel}
\usepackage[colorlinks]{hyperref}
\usepackage{bookmark}

% logos
\makeatletter
\newcommand\eTeX{$\m@th\varepsilon$-\TeX}
\newcommand\LuaTeX{Lua\TeX}
\renewcommand\PlainTeX{Plain\thinspace\TeX}
\newcommand\TeXe{\TeX\thinspace82}
\newcommand\TeXLive{\TeX\thinspace Live}
\makeatother

% logos for the lazy me (mpg)
\newcommand\tex{\TeX\xspace}
\newcommand\latex{\LaTeX\xspace}
\newcommand\etex{\eTeX\xspace}
\newcommand\luatex{\LuaTeX\xspace}
\newcommand\texe{\TeXe\xspace}
\newcommand\texlive{\TeXLive\xspace}
\newcommand\plaintex{\PlainTeX\xspace}

% special elements of text
\newcommand\file{\nolinkurl}
\newcommand\email[1]{\href{mailto:#1}{#1}}
\newcommand\pk{\textsf}
\newcommand\cmdname{\texttt}

% for hyperref
\pdfstringdefDisableCommands{%
  \def\cs#1{\@backslashchar #1}%
  }

% easy verbatim
\MakeShortVerb\|

\begin{document}
  \DocInput{luatexbase-modutils.dtx}%
\end{document}
%</driver>
% \fi
%
% \CheckSum{0}
%
% \CharacterTable
%  {Upper-case    \A\B\C\D\E\F\G\H\I\J\K\L\M\N\O\P\Q\R\S\T\U\V\W\X\Y\Z
%   Lower-case    \a\b\c\d\e\f\g\h\i\j\k\l\m\n\o\p\q\r\s\t\u\v\w\x\y\z
%   Digits        \0\1\2\3\4\5\6\7\8\9
%   Exclamation   \!     Double quote  \"     Hash (number) \#
%   Dollar        \$     Percent       \%     Ampersand     \&
%   Acute accent  \'     Left paren    \(     Right paren   \)
%   Asterisk      \*     Plus          \+     Comma         \,
%   Minus         \-     Point         \.     Solidus       \/
%   Colon         \:     Semicolon     \;     Less than     \<
%   Equals        \=     Greater than  \>     Question mark \?
%   Commercial at \@     Left bracket  \[     Backslash     \\
%   Right bracket \]     Circumflex    \^     Underscore    \_
%   Grave accent  \`     Left brace    \{     Vertical bar  \|
%   Right brace   \}     Tilde         \~}
%
% \title{The \pk{luatexbase-modutils} package}
% \date{v0.2 2010-05-12}
% \author{%
%  Manuel P\'egouri\'e-Gonnard \\ \email{mpg@elzevir.fr} \and
%  \'Elie Roux \\ \email{elie.roux@telecom-bretagne.eu}}
%
% \maketitle
%
% \begin{abstract}
% This package provides functions similar to \latex's |\usepackage| and
% |\ProvidesPackage| macros,\footnote{and their variants or synonyms such as
% \cs{documentclass} and \cs{RequirePackage} or \cs{ProvidesClass} and
% \cs{ProvidesFiles}} or more precisely the part of these macros that deals
% with identification and version checking (no attempt is done at implementing
% an option mechanism). Functions for error reporting are provided too.
%
% It also loads \pk{luatexbase-loader}.
% \end{abstract}
%
% \tableofcontents
%
% \section{Documentation}
%
% Lua's standard function |require()| is similar to \tex's |\input| primitive
% but is somehow more evolved in that it makes a few checks to avoid loading
% the same module twice. In the \tex world, this needs to be taken care of by
% macro packages; in the \latex world this is done by |\usepackage|.
%
% But |\usepackage| also takes care of many other things. Most notably, it
% implements a complex option system, and does some identification and version
% checking. The present package doesn't try to provide anything for options,
% but implements a system for identification and version checking similar to
% \latex's system. Both \tex macros and Lua functions are provided.
%
% This package also provides Lua functions for reporting errors, warnings,
% etc.
%
% It is important to notice that Lua's standard function |module()| is
% completely unrelated to the present package. It has nothing to do with
% identification and deals only with namespaces.\footnote{More precisely, it
% modifies the current environment.} So, you should continue to
% use it normally, unlike the |require()| function which can be replaced with
% this package's |luatexbase.require_module()|.
%
% \subsection{\tex macros}
%
% The two macros |\luatexUseModule| and |\luatexRequireModule| are very
% similar and are interfaces to the Lua functions |use_module| and
% |require_module|. The only difference between those macros is the number of
% arguments (just as the underlying Lua functions): |\luatexUseModule| only
% take one argument: the module name\footnote{without extension} while
% |\luatexRequireModule| takes another argument for specifying a minimal
% version (see below). With \latex, this argument is the first and is
% optional. Otherwise, it's the second one and it's mandatory.
%
% \subsection{Lua functions}
%
% The main functions are |luatexbase.require_module| and
% |luatexbase.use_module| which may be used as a replacement to |require()|.
% The only difference between these functions is, |require_module| accepts a
% second, optional argument in order to specify a minimal version. They do the
% same as |require()| but also make sure the module loaded correctly
% identifies itself with the name given, and its version is greater than the
% minimal version required. The version can be given either as a (floating
% point) number or as a date in YYYY/MM/DD format.
%
% Modules identify themselves using |luatexbase.provides_module|, whose only
% argument is a table with some information about the module. The mandatory
% fields are |name| (a string), |version| (a number), |date| (a string) and
% |description| (a string). Other fields are optional and ignored, and usually
% include |copyright|, |author| and |license|.
%
% \bigskip
%
% Functions for reporting are provided; similarly to \latex's |\PackageError|
% etc. they take the module name as their first argument and include it in the
% printed message in an appropriate way. The remaining arguments are passed to
% |string.format()| before being printed.
%
% The functions provided (all found in the |luatexbase| table) are
% |module_error|, |module_warning|, |module_info| (writes to terminal and
% log), |module_log| (writes only to the log file) and |module_term| (writes
% only to the terminal).
%
%    \section{Implementation}
%
%    \subsection{\tex package}
%
%    \begin{macrocode}
%<*texpackage>
%    \end{macrocode}
%
%    \subsubsection{Preliminaries}
%
%    Reload protection, especially for \plaintex.
%
%    \begin{macrocode}
                \csname lltxb@modutils@loaded\endcsname
\expandafter\let\csname lltxb@modutils@loaded\endcsname\endinput
%    \end{macrocode}
%
%    Catcode defenses.
%
%    \begin{macrocode}
\begingroup
  \catcode123 1 % {
  \catcode125 2 % }
  \catcode 35 6 % #
  \toks0{}%
  \def\x{}%
  \def\y#1 #2 {%
    \toks0\expandafter{\the\toks0 \catcode#1 \the\catcode#1}%
    \edef\x{\x \catcode#1 #2}}%
  \y 123 1  % {
  \y 125 2  % }
  \y  35 6  % #
  \y  10 12 % ^^J
  \y  34 12 % "
  \y  36 3  % $ $
  \y  39 12 % '
  \y  40 12 % (
  \y  41 12 % )
  \y  42 12 % *
  \y  43 12 % +
  \y  44 12 % ,
  \y  45 12 % -
  \y  46 12 % .
  \y  47 12 % /
  \y  60 12 % <
  \y  61 12 % =
  \y  64 11 % @ (letter)
  \y  62 12 % >
  \y  95 12 % _ (other)
  \y  96 12 % `
  \edef\y#1{\endgroup\edef#1{\the\toks0\relax}\x}%
\expandafter\y\csname lltxb@modutils@AtEnd\endcsname
%    \end{macrocode}
%
%    Package declaration.
%
%    \begin{macrocode}
\begingroup
  \expandafter\ifx\csname ProvidesPackage\endcsname\relax
    \def\x#1[#2]{\immediate\write16{Package: #1 #2}}
  \else
    \let\x\ProvidesPackage
  \fi
\expandafter\endgroup
\x{luatexbase-modutils}[2010/05/27 v0.2a Module utilities for LuaTeX]
%    \end{macrocode}
%
%    Make sure \luatex is used.
%
%    \begin{macrocode}
\begingroup\expandafter\expandafter\expandafter\endgroup
\expandafter\ifx\csname RequirePackage\endcsname\relax
  \input ifluatex.sty
\else
  \RequirePackage{ifluatex}
\fi
\ifluatex\else
  \begingroup
    \expandafter\ifx\csname PackageWarningNoLine\endcsname\relax
      \def\x#1#2{\begingroup\newlinechar10
        \immediate\write16{Package #1 warning: #2}\endgroup}
    \else
      \let\x\PackageWarningNoLine
    \fi
  \expandafter\endgroup
  \x{luatexbase-modutils}{LuaTeX is required for this package. Aborting.}
  \lltxb@modutils@AtEnd
  \expandafter\endinput
\fi
%    \end{macrocode}
%
%    Load \pk{luatexbase-loader} (hence \pk{luatexbase-compat}) and require
%    supporting Lua module.
%
%    \begin{macrocode}
\begingroup\expandafter\expandafter\expandafter\endgroup
\expandafter\ifx\csname RequirePackage\endcsname\relax
  \input luatexbase-loader.sty
\else
  \RequirePackage{luatexbase-loader}
\fi
\luatexbase@directlua{require('luatexbase.modutils')}
%    \end{macrocode}
%
%    Make sure the primitives we need are available.
%
%    \begin{macrocode}
\luatexbase@ensure@primitive{luaescapestring}
%    \end{macrocode}
%
%    \subsubsection{User macros}
%
%    Interface to |use_module()|.
%
%    \begin{macrocode}
\def\luatexUseModule#1{\luatexbase@directlua{%
    luatexbase.use_module("\luatexluaescapestring{#1}")}}
%    \end{macrocode}
%
%    Interface to |require_module()| with syntax depending on the format.
%
%    \begin{macrocode}
\begingroup\expandafter\expandafter\expandafter\endgroup
\expandafter\ifx\csname newcommand\endcsname\relax
  \def\luatexRequireModule#1#2{%
    \luatexbase@directlua{luatexbase.require_module(
      "\luatexluaescapestring{#1}", "\luatexluaescapestring{#2}")}}
\else
  \newcommand\luatexRequireModule[2][0]{%
    \luatexbase@directlua{luatexbase.require_module(
      "\luatexluaescapestring{#2}", "\luatexluaescapestring{#1}")}}
\fi
%    \end{macrocode}
%
%    \begin{macrocode}
\lltxb@modutils@AtEnd
%</texpackage>
%    \end{macrocode}
%
%    \subsection{Lua module}
%
%    \begin{macrocode}
%<*luamodule>
module("luatexbase", package.seeall)
%    \end{macrocode}
%
%    \subsection{Internal functions and data}
%
%    Tables holding informations about the modules loaded and the versions
%    required.
%
%    \begin{macrocode}
local modules = modules or {}
local requiredversions = {}
%    \end{macrocode}
%
%    Convert a date in YYYY/MM/DD format into a number
%
%    \begin{macrocode}
local function datetonumber(date)
    numbers = string.gsub(date, "(%d+)/(%d+)/(%d+)", "%1%2%3")
    return tonumber(numbers)
end
%    \end{macrocode}
%
%    Say if a string is a date in YYYY//MM/DD format.
%
%    \begin{macrocode}
local function isdate(date)
    for _, _ in string.gmatch(date, "%d+/%d+/%d+") do
        return true
    end
    return false
end
%    \end{macrocode}
%
%    Parse a version into a table indicating a type (date or number), a
%    numeric version and the original version string.
%
%    \begin{macrocode}
local date, number = 1, 2
local function parse_version(version)
    if isdate(version) then
        return {type = date, version = datetonumber(version), orig = version}
    else
        return {type = number, version = tonumber(version), orig = version}
    end
end
%    \end{macrocode}
%
%    \subsubsection{Error, warning and info function for modules}
%
%    Here are the reporting functions for the modules. For errors, Lua's
%    |error()| is used. For now, the error reports look less good than with
%    \tex's |\errmessage|, but hopefully it will be improved in future
%    versions of \luatex. We could invoke |\errmessage| using |tex.sprint()|,
%    but it may cause problems on the \tex end, and moreover |error()| will
%    still be used by Lua for other errors, so it makes messages more
%    consistent.
%
%    \begin{macrocode}
local function module_error_int(mod, ...)
  error('Module '..mod..' error: '..string.format(...), 3)
end
function module_error(mod, ...)
  module_error_int(mod, ...)
end
function module_warning(mod, ...)
  texio.write_nl("Module "..mod.." warning: "..string.format(...))
end
function module_info(mod, ...)
  texio.write_nl(mod..": "..string.format(...))
end
function module_log(mod, ...)
  texio.write_nl('log', mod..": "..string.format(...))
end
function module_term(mod, ...)
  texio.write_nl('term', mod..": "..string.format(...))
end
%    \end{macrocode}
%
%    For our own convenience, local functions for warning and errors in the
%    present module.
%
%    \begin{macrocode}
local function err(...) module_error_int('luatexbase.modutils', ...) end
local function warn(...) module_warning('luatexbase.modutils', ...) end
%    \end{macrocode}
%
%    \subsubsection{module loading and providing functions}
%
%    Load a module without version checking.
%
%    \begin{macrocode}
function use_module(name)
    require(name)
    if not modules[name] then
        warn("Module didn't properly identified itself: %s", name)
    end
end
%    \end{macrocode}
%
%    Load a module with optional version checking.
%
%    \begin{macrocode}
function require_module(name, version)
    if not version then
        return use_module(name)
    end
    luaversion = parse_version(version)
    if modules[name] then
        if luaversion.type == date then
            if datetonumber(modules[name].date) < luaversion.version then
                err("found module `%s' loaded in version %s, "
                .."but version %s was required",
                name, modules[name].date, version)
            end
        else
            if modules[name].version < luaversion.version then
                err("found module `%s' loaded in version %.02f, "
                .."but version %s was required",
                name, modules[name].version, version)
            end
        end
    else
        requiredversions[name] = luaversion
        use_module(name)
    end
end
%    \end{macrocode}
%
%    Provide identification information for a module.
%
%    \begin{macrocode}
function provides_module(mod)
    if not mod then
        err('cannot provide nil module')
        return
    end
    if not mod.version or not mod.name or not mod.date
    or not mod.description then
        err("invalid module registered: "
        .."fields name, version, date and description are mandatory")
        return
    end
    requiredversion = requiredversions[mod.name]
    if requiredversion then
        if requiredversion.type == date
        and requiredversion.version > datetonumber(mod.date) then
            err("loading module %s in version %s, "
            .."but version %s was required",
            mod.name, mod.date, requiredversion.orig)
        elseif requiredversion.type == number
        and requiredversion.version > mod.version then
            err("loading module %s in version %.02f, "
            .."but version %s was required",
            mod.name, mod.version, requiredversion.orig)
        end
    end
    modules[mod.name] = mod
    texio.write_nl('log', string.format("Lua module: %s %s v%.02f %s\n",
    mod.name, mod.date, mod.version, mod.description))
end
%    \end{macrocode}
%
%    \begin{macrocode}
%</luamodule>
%    \end{macrocode}
%
%    \section{Test files}
%
%    A dummy lua file for tests.
%
%    \begin{macrocode}
%<*testdummy>
luatexbase.provides_module {
  name        = 'test-modutils',
  date        = '2000/01/01',
  version     = 1,
  description = 'dummy test package',
}
%</testdummy>
%    \end{macrocode}
%
%    We just check that the package loads properly, under both LaTeX and Plain
%    TeX. Anyway, the test files of other modules using this one already are a
%    test\dots
%
%    \begin{macrocode}
%<testplain>\input luatexbase-modutils.sty
%<testlatex>\RequirePackage{luatexbase-modutils}
%<*testplain,testlatex>
\luatexUseModule{test-modutils}
\luatexRequireModule
%<testlatex>[1970/01/01]
{test-modutils}
%<testplain>{1970/01/01}
%</testplain,testlatex>
%<testplain>\bye
%<testlatex>\stop
%    \end{macrocode}
%
% \Finale
\endinput
